\documentclass[11pt,a4paper]{article}
\usepackage[utf8]{inputenc}
\usepackage[T1]{fontenc}
\usepackage{geometry}
\usepackage{titlesec}
\usepackage{listings}
\usepackage{hyperref}
\geometry{margin=1in}
\titleformat{\section}{\large\bfseries}{\thesection}{0.5em}{}
\titleformat{\subsection}{\normalsize\bfseries}{\thesubsection}{0.5em}{}
\titleformat{\subsubsection}{\normalsize\bfseries}{\thesubsubsection}{0.5em}{}

\lstset{
  basicstyle=\ttfamily\small,
  keywordstyle=\bfseries,
  commentstyle=\itshape,
  breaklines=true,
  columns=fullflexible,
  frame=single,
  numbers=left,
  numberstyle=\tiny,
  stepnumber=1,
  tabsize=2,
  showstringspaces=false
}

\title{Neurobloom Simulation -- Technická dokumentácia (LaTeX)}
\author{Codex (ChatGPT)}
\date{2025-12-28}

\begin{document}
\maketitle

\section{Meta}
\begin{itemize}
  \item Modul: Simulation (interaktívna simulácia neurónov)
  \item Verzia: 0.1
  \item Rozsah: \texttt{SimulationPage} + UI panely + 3D scéna + simulačné funkcie + algoritmy
\end{itemize}

\section{Prehľad architektúry}
Simulation stránka spája 3D vizualizáciu siete, UI panely pre ovládanie a
``živú'' simulačnú logiku. Hlavné vrstvy:
\begin{itemize}
  \item \texttt{src/pages/SimulationPage.tsx} -- layout, navigácia a prepojenie panelov.
  \item \texttt{src/hooks/useNeuralNetwork.ts} -- stav siete, tréning, algoritmy a štatistiky.
  \item \texttt{src/components/three/NeuralNetworkScene.tsx} + \texttt{Neuron.tsx}, \texttt{Connection.tsx} -- 3D scéna.
  \item \texttt{src/components/ui/*} -- ovládacie panely, štatistiky, overlay.
  \item \texttt{src/simulation/*} + \texttt{src/algorithms/*} -- dátové typy a algoritmy.
\end{itemize}

\section{SimulationPage (src/pages/SimulationPage.tsx)}

\subsection{Členenie zdrojového textu a odporúčané komentovanie}
\paragraph{Hlavička súboru}
\begin{lstlisting}[language=JavaScript, caption={Odporúčaná hlavička súboru \texttt{SimulationPage.tsx}}, label={lst:sim-header}]
/*
============================================================
Autor:        <Meno autora / tím>
Dátum:        2025-12-28
Verzia:       1.0
Súbor:        src/pages/SimulationPage.tsx
Názov:        SimulationPage
Popis:        Hlavná stránka simulácie neurónov.
              Spája 3D scénu, UI panely a simulačný hook.
============================================================
*/
\end{lstlisting}

\subsection{Napojenie na hook a správa stavu}
\paragraph{Funkcia}
Stránka získava všetky potrebné callbacky a stav z \texttt{useNeuralNetwork}.
Lokálne si udržiava \texttt{selectedNeuronId} pre výber neurónu v 3D scéne.

\begin{lstlisting}[language=TypeScript, caption={Napojenie na \texttt{useNeuralNetwork}}, label={lst:sim-hook}]
const [selectedNeuronId, setSelectedNeuronId] = useState<string | null>(null);
const sceneRef = useRef<NeuralNetworkSceneHandle>(null);

const {
  neurons, mode, stats, addNeuron, addMultipleNeurons,
  startTraining, stopTraining, resetNetwork, initializeNetwork,
  runAlgorithm, stopAlgorithm, isAlgorithmRunning,
  currentAlgorithm, algorithmProgress, neuronsCreated,
  activationFocus, currentPattern, currentProcessingNeuron,
  algorithmSpeed, setAlgorithmSpeed,
} = useNeuralNetwork();
\end{lstlisting}

\subsection{Inicializácia siete}
\paragraph{Funkcia}
Pri prvom renderi, ak sieť neobsahuje žiadne neuróny, zavolá sa
\texttt{initializeNetwork} (vytvorí sa jeden vstupný neurón).

\begin{lstlisting}[language=TypeScript, caption={Inicializácia siete}, label={lst:sim-init}]
useEffect(() => {
  if (neurons.length === 0) {
    initializeNetwork();
  }
}, [neurons.length, initializeNetwork]);
\end{lstlisting}

\subsection{Ošetrenie výberu neurónu}
\paragraph{Funkcia}
Ak zvolený neurón už neexistuje (napr. po resete), výber sa zruší.

\begin{lstlisting}[language=TypeScript, caption={Kontrola platnosti \texttt{selectedNeuronId}}, label={lst:sim-selection}]
useEffect(() => {
  if (selectedNeuronId && !neurons.find((n) => n.id === selectedNeuronId)) {
    setSelectedNeuronId(null);
  }
}, [neurons, selectedNeuronId]);
\end{lstlisting}

\subsection{Rozloženie UI}
\paragraph{Funkcia}
Hlavný layout používa tri stĺpce: ľavý (algoritmy + ovládanie), stredný
(3D scéna + overlay), pravý (štatistiky + referencie neurónov).

\begin{lstlisting}[language=TypeScript, caption={Rozloženie panelov}, label={lst:sim-layout}]
<div className="grid grid-cols-1 lg:grid-cols-[260px_1fr_260px] gap-4">
  <div className="panel p-4 space-y-4">
    <AlgorithmPanel ... />
    <ControlPanel ... />
  </div>

  <div className="h-full panel overflow-hidden relative">
    <NeuralNetworkScene ... />
    <AlgorithmInfoOverlay ... />
  </div>

  <div className="space-y-4">
    <div className="panel p-4 space-y-4">
      <StatsDisplay stats={stats} />
      <NeuronReferencePanel ... />
    </div>
  </div>
</div>
\end{lstlisting}

\section{Hook useNeuralNetwork (src/hooks/useNeuralNetwork.ts)}

\subsection{Členenie a odporúčané komentovanie}
\paragraph{Hlavička súboru}
\begin{lstlisting}[language=JavaScript, caption={Odporúčaná hlavička \texttt{useNeuralNetwork.ts}}, label={lst:hook-header}]
/*
============================================================
Autor:        <Meno autora / tím>
Dátum:        2025-12-28
Verzia:       1.0
Súbor:        src/hooks/useNeuralNetwork.ts
Názov:        useNeuralNetwork
Popis:        React hook pre stav siete, tréning, algoritmy a štatistiky.
============================================================
*/
\end{lstlisting}

\subsection{Základný stav siete}
\paragraph{Funkcia}
Hook drží pole neurónov, režim simulácie a štatistiky siete.
\begin{lstlisting}[language=TypeScript, caption={Stav siete}, label={lst:hook-state}]
const [neurons, setNeurons] = useState<Neuron[]>([]);
const [mode, setMode] = useState<SimulationMode>("idle");
const [stats, setStats] = useState<NetworkStats>({
  totalNeurons: 0,
  totalConnections: 0,
  averageActivation: 0,
  averageHealth: 1,
  trainingEpochs: 0,
  accuracy: 0,
  isOverfitted: false,
  isUnderfitted: false,
});
\end{lstlisting}

\subsection{Pridávanie neurónov a prepojení}
\paragraph{Funkcia}
\texttt{addNeuron} vytvorí nový neurón v náhodnej pozícii a spojí ho
s niekoľkými existujúcimi neurónmi.

\begin{lstlisting}[language=TypeScript, caption={Pridanie neurónu}, label={lst:hook-add-neuron}]
const addNeuron = useCallback((type: "input" | "hidden" | "output" = "hidden") => {
  const position = randomSpherePosition(3 + neurons.length * 0.05);
  const newNeuron = createNeuron(position, type);
  const connections: Connection[] = [];
  const numConnections = Math.min(2, neurons.length);
  for (let i = 0; i < numConnections; i++) {
    const randomNeuron = neurons[Math.floor(Math.random() * neurons.length)];
    if (randomNeuron) connections.push(createConnection(randomNeuron.id, newNeuron.id));
  }
  newNeuron.connections = connections;
  setNeurons((prev) => [...prev, newNeuron]);
}, [neurons]);
\end{lstlisting}

\subsection{Tréningová slučka}
\paragraph{Funkcia}
Tréning beží cez \texttt{setInterval} (400 ms). Strieda dva vstupné vzory,
aktivuje neuróny, aktualizuje váhy Hebbovým pravidlom a vypočítava štatistiky.

\begin{lstlisting}[language=TypeScript, caption={Spustenie tréningu}, label={lst:hook-training}]
const startTraining = useCallback(() => {
  setMode("training");
  if (trainingInterval.current) clearInterval(trainingInterval.current);
  let tick = 0;
  trainingInterval.current = setInterval(() => {
    tick++;
    const patternPhase = Math.floor(tick / 20) % 2;
    const patternName = patternPhase === 0 ? "Pattern A (Even Inputs)" : "Pattern B (Odd Inputs)";
    setCurrentPattern(prev => prev !== patternName ? patternName : prev);

    setNeurons((prev) => {
      // 1) aktualizuj vstupy
      // 2) aktivuj skryté/výstupné neuróny
      // 3) uprav váhy spojení
      return [...updatedInputs, ...updatedHiddenOutput];
    });
  }, 400);
}, []);
\end{lstlisting}

\subsection{Algoritmy (AlgorithmRunner)}
\paragraph{Funkcia}
\texttt{runAlgorithm} pripraví minimálny počet neurónov (40), postupne ich
pridáva do sféry a potom spúšťa algoritmus v \texttt{AlgorithmRunner}.

\begin{lstlisting}[language=TypeScript, caption={Spustenie algoritmu}, label={lst:hook-run-algorithm}]
const runAlgorithm = useCallback((algorithmType: AlgorithmType) => {
  setMode("idle");
  if (trainingInterval.current) clearInterval(trainingInterval.current);

  setIsAlgorithmRunning(true);
  setCurrentAlgorithm(algorithmType);
  setAlgorithmProgress(0);
  setNeuronsCreated(0);
  setIsStatsLocked(true);

  const minNeurons = 40;
  if (neurons.length < minNeurons) {
    // postupné pridávanie neurónov + prepojenia
  } else {
    if (!algorithmRunner.current) algorithmRunner.current = new AlgorithmRunner(neurons);
    algorithmRunner.current.start(algorithmType);
  }
}, [neurons]);
\end{lstlisting}

\subsection{Štatistiky a fokus}
\paragraph{Funkcia}
Štatistiky (počet neurónov, priemerné aktivácie, zdravie) sa počítajú
len ak nie sú zamknuté alebo ak beží algoritmus. Fokus vyberá neurón
s najvyššou aktiváciou pre UI panel.

\begin{lstlisting}[language=TypeScript, caption={Výpočet štatistík + live focus}, label={lst:hook-stats}]
useEffect(() => {
  if (neurons.length > 0 && (!isStatsLocked || isAlgorithmRunning)) {
    const totalConnections = neurons.reduce((sum, n) => sum + n.connections.length, 0);
    const avgActivation = neurons.reduce((sum, n) => sum + n.activation, 0) / neurons.length;
    const avgHealth = neurons.reduce((sum, n) => sum + n.health, 0) / neurons.length;
    setStats((prev) => ({ ...prev, totalNeurons: neurons.length, totalConnections, averageActivation: avgActivation, averageHealth: avgHealth }));
  }
}, [neurons, isStatsLocked, isAlgorithmRunning]);
\end{lstlisting}

\section{Simulačné dátové typy a funkcie}

\subsection{Dátové typy (src/simulation/types.ts)}
\paragraph{Funkcia}
Typy definujú neurón, spojenie a štatistiky siete. Typ \texttt{SimulationMode}
určuje režim (idle, training, inference, degrading).

\begin{lstlisting}[language=TypeScript, caption={Typy neurónu a spojenia}, label={lst:sim-types}]
export interface Neuron {
  id: string;
  position: THREE.Vector3;
  activation: number; // 0-1
  age: number;
  connections: Connection[];
  learningRate: number;
  health: number;
  trainingCount: number;
  type: "input" | "hidden" | "output";
  color: THREE.Color;
}

export interface Connection {
  id: string;
  from: string;
  to: string;
  weight: number; // -1..1
  strength: number; // 0..1
  age: number;
  lastActivation: number;
}
\end{lstlisting}

\subsection{Simulačné funkcie (src/simulation/neuralNetwork.ts)}
\paragraph{Funkcia}
Súbor obsahuje tvorbu neurónov/spojení, aktivačné funkcie a pravidlá učenia.

\begin{lstlisting}[language=TypeScript, caption={Vytvorenie neurónu + spojenia}, label={lst:sim-create}]
export const createNeuron = (position: THREE.Vector3, type: "input" | "hidden" | "output" = "hidden"): Neuron => {
  const colorMap = {
    input: new THREE.Color("#00D4FF"),
    hidden: new THREE.Color("#B565FF"),
    output: new THREE.Color("#00FF88"),
  };
  return {
    id: `neuron_${Date.now()}_${Math.random().toString(36).substr(2, 9)}`,
    position: position.clone(),
    activation: 0,
    age: 0,
    connections: [],
    learningRate: 0.1,
    health: 1.0,
    trainingCount: 0,
    type,
    color: colorMap[type],
  };
};
\end{lstlisting}

\begin{lstlisting}[language=TypeScript, caption={Aktivácia neurónu}, label={lst:sim-activate}]
export const activateNeuron = (neuron: Neuron, inputs: Map<string, number>): number => {
  let sum = 0;
  neuron.connections.forEach((conn) => {
    const inputValue = inputs.get(conn.from) || 0;
    sum += inputValue * conn.weight;
  });
  return sigmoid(sum);
};
\end{lstlisting}

\begin{lstlisting}[language=TypeScript, caption={Hebb/Oja pravidlo pre úpravu váh}, label={lst:sim-hebb}]
export const updateConnectionWeight = (
  connection: Connection,
  sourceActivation: number,
  targetActivation: number,
  learningRate: number
): number => {
  const hebbianTerm = sourceActivation * targetActivation;
  const decayTerm = 0.05 * connection.weight * targetActivation * targetActivation;
  const delta = learningRate * (hebbianTerm - decayTerm);
  const newWeight = connection.weight + delta;
  return Math.max(-1, Math.min(1, newWeight));
};
\end{lstlisting}

\begin{lstlisting}[language=TypeScript, caption={Zdravie neurónu a detekcia problémov}, label={lst:sim-health}]
export const calculateNeuronHealth = (neuron: Neuron): number => {
  const ageDecay = Math.max(0, 1 - neuron.age / 300);
  const overtrainingPenalty = Math.max(0, 1 - neuron.trainingCount / 10000);
  return Math.min(1, (ageDecay + overtrainingPenalty) / 2 * neuron.health);
};

export const detectTrainingIssues = (trainingAccuracy: number, validationAccuracy: number, epochs: number) => {
  const gap = trainingAccuracy - validationAccuracy;
  return {
    isOverfitted: gap > 0.15 && epochs > 100,
    isUnderfitted: trainingAccuracy < 0.7 && epochs > 50,
  };
};
\end{lstlisting}

\subsection{Re-exporty (src/simulation/index.ts)}
\paragraph{Funkcia}
Súbor zjednodušuje importy tým, že re-exportuje typy a funkcie.

\begin{lstlisting}[language=TypeScript, caption={Re-export simulácie}, label={lst:sim-index}]
export * from "./types";
export * from "./neuralNetwork";
\end{lstlisting}

\section{Algoritmy vizualizácie}

\subsection{Definície algoritmov (src/algorithms/types.ts)}
\paragraph{Funkcia}
Definície obsahujú ID algoritmu, názov, popis, trvanie a farbu.

\begin{lstlisting}[language=TypeScript, caption={ALGORITHMS -- metadáta}, label={lst:algorithms-types}]
export type AlgorithmType =
  | 'wave-propagation'
  | 'spiral-growth'
  | 'cascade-activation'
  | 'pulse-network'
  | 'random-walker';

export const ALGORITHMS: Algorithm[] = [
  { id: 'wave-propagation', name: 'Wave Propagation', duration: 9, color: '#00D4FF', ... },
  { id: 'spiral-growth', name: 'Spiral Growth', duration: 11, color: '#B565FF', ... },
  ...
];
\end{lstlisting}

\subsection{AlgorithmRunner (src/algorithms/algorithmRunner.ts)}
\paragraph{Funkcia}
Trieda riadi farebné a aktivačné animácie. Drží pôvodné farby neurónov,
spúšťa algoritmy a poskytuje odkaz na ``aktuálne spracovávaný'' neurón.

\begin{lstlisting}[language=TypeScript, caption={Lifecycle algoritmu}, label={lst:algorithm-runner}]
start(algorithmType: AlgorithmType) {
  this.currentAlgorithm = algorithmType;
  this.startTime = Date.now();
  this.isRunning = true;
  this.originalColors.clear();
  this.neurons.forEach(neuron => {
    this.originalColors.set(neuron.id, neuron.color.clone());
  });
}

stop() {
  this.isRunning = false;
  this.currentAlgorithm = null;
  this.neurons.forEach(neuron => {
    neuron.activation = 0;
    const originalColor = this.originalColors.get(neuron.id);
    if (originalColor) neuron.color.copy(originalColor);
  });
  this.originalColors.clear();
}
\end{lstlisting}

\paragraph{Prehľad algoritmov}
\begin{itemize}
  \item \texttt{wave-propagation} -- radiálna vlna aktivácie od stredu, neuróny sa sfarbujú do modrej.
  \item \texttt{spiral-growth} -- rotujúci špirálový vzor, plynulé farebné prechody.
  \item \texttt{cascade-activation} -- postupná aktivácia neurónov zoradených podľa pozície.
  \item \texttt{pulse-network} -- globálny pulz siete s rytmickým prechodom farieb.
  \item \texttt{random-walker} -- náhodný ``skokan'' aktivuje lokálne okolie.
\end{itemize}

\section{3D vizualizácia simulácie}

\subsection{NeuralNetworkScene (src/components/three/NeuralNetworkScene.tsx)}
\paragraph{Funkcia}
Hlavný 3D komponent vytvára scénu, osvetlenie, pozadie, spája neuróny,
vykresľuje spojenia a častice toku dát. Podporuje reset kamery.

\paragraph{Hlavička súboru}
\begin{lstlisting}[language=JavaScript, caption={Odporúčaná hlavička \texttt{NeuralNetworkScene.tsx}}, label={lst:scene-header}]
/*
============================================================
Autor:        <Meno autora / tím>
Dátum:        2025-12-28
Verzia:       1.0
Súbor:        src/components/three/NeuralNetworkScene.tsx
Názov:        NeuralNetworkScene
Popis:        Hlavná 3D scéna simulácie neurónovej siete.
============================================================
*/
\end{lstlisting}

\subsubsection{DataFlowParticles}
\paragraph{Funkcia}
Častice putujú po aktívnych spojeniach. Rýchlosť a veľkosť závisí
od váhy a aktivácie zdrojového neurónu.

\begin{lstlisting}[language=TypeScript, caption={DataFlowParticles}, label={lst:scene-particles}]
const DataFlowParticles = ({ neurons }: { neurons: NeuronType[] }) => {
  const activeConnections = useMemo(() => {
    return neurons.flatMap(n => n.connections.map(c => ({
      ...c,
      start: n.position,
      end: neurons.find(target => target.id === c.to)?.position,
      sourceActivation: n.activation,
    }))).filter(c => c.end);
  }, [neurons]);
  ...
};
\end{lstlisting}

\subsubsection{GradientBackground}
\paragraph{Funkcia}
Pozadie je shaderMaterial s jemným vertikálnym a radiálnym gradientom.

\begin{lstlisting}[language=TypeScript, caption={GradientBackground}, label={lst:scene-gradient}]
const GradientBackground = () => {
  const { camera } = useThree();
  const planeRef = useRef<THREE.Mesh>(null);
  useFrame(() => {
    if (!planeRef.current) return;
    camera.getWorldDirection(dir);
    planeRef.current.position.copy(camera.position).addScaledVector(dir, 60);
    planeRef.current.quaternion.copy(camera.quaternion);
  });
  return (
    <mesh ref={planeRef} scale={140} frustumCulled={false} renderOrder={-10}>
      <planeGeometry args={[1, 1]} />
      <shaderMaterial ... />
    </mesh>
  );
};
\end{lstlisting}

\subsubsection{CameraTracker a ScreenReference}
\paragraph{Funkcia}
\texttt{CameraTracker} sleduje yaw/pitch/distance kamery a posiela ich do HUD.
\texttt{ScreenReference} zobrazí orientačnú kocku a tlačidlo reset.

\begin{lstlisting}[language=TypeScript, caption={CameraTracker + HUD}, label={lst:scene-camera}]
const CameraTracker = ({ onChange }: { onChange: (data: HudData) => void }) => {
  const { camera } = useThree();
  useFrame(() => {
    const pos = camera.position;
    const yawDisplay = Math.atan2(pos.x, pos.z) * THREE.MathUtils.RAD2DEG;
    const pitch = Math.atan2(pos.y, Math.sqrt(pos.x * pos.x + pos.z * pos.z)) * THREE.MathUtils.RAD2DEG;
    onChange({ yawDisplay, yawContinuous, pitch, distance: pos.length() });
  });
  return null;
};
\end{lstlisting}

\subsubsection{Neurón a spojenia}
\paragraph{Funkcia}
\texttt{Neuron} je pulzujúca sféra s glow efektmi. \texttt{Connection}
vykresľuje \texttt{THREE.Line} medzi neurónmi a upravuje opacity podľa aktivácie.

\begin{lstlisting}[language=TypeScript, caption={Neuron (pulz a glow)}, label={lst:scene-neuron}]
useFrame((state) => {
  const pulse = 1 + neuron.activation * 0.6;
  meshRef.current.scale.setScalar(pulse * 0.25 * highlightBoost);
  material.color.copy(targetColor);
  material.emissive.copy(targetColor);
});
\end{lstlisting}

\begin{lstlisting}[language=TypeScript, caption={Connection (line opacity + color)}, label={lst:scene-connection}]
useFrame((state) => {
  const avgActivation = (fromNeuron.activation + toNeuron.activation) / 2;
  const pulse = Math.sin(state.clock.elapsedTime * 8) * 0.5 + 0.5;
  material.opacity = 0.3 + avgActivation * connection.strength * 0.5 + pulse * avgActivation * 0.4;
  const baseColor = connection.weight > 0 ? new THREE.Color("#00D4FF") : new THREE.Color("#B565FF");
  material.color.copy(baseColor.multiplyScalar(1 + avgActivation * 0.5));
});
\end{lstlisting}

\subsection{Hlavná scéna}
\paragraph{Funkcia}
\texttt{NeuralNetworkScene} skladá svetlá, hviezdy, sparkles, neuróny,
spojenia, častice a \texttt{OrbitControls}.

\begin{lstlisting}[language=TypeScript, caption={Canvas a obsah scény}, label={lst:scene-canvas}]
<Canvas gl={{ antialias: true, powerPreference: "high-performance", alpha: false }} dpr={[1, 2]}>
  <color attach="background" args={["#050510"]} />
  <PerspectiveCamera ref={cameraRef} makeDefault position={[0, 0, 15]} fov={60} />
  <GradientBackground />
  <ReferenceGrid />
  <ambientLight intensity={0.7} color="#99BBFF" />
  <pointLight position={[10, 10, 10]} intensity={3.5} color="#00D4FF" />
  <Stars radius={100} depth={50} count={800} factor={2} fade />
  {neurons.map(neuron => <Neuron ... />)}
  {allConnections.map(conn => <Connection ... />)}
  <DataFlowParticles neurons={neurons} />
  <OrbitControls ref={controlsRef} enablePan enableZoom enableRotate />
</Canvas>
\end{lstlisting}

\section{UI komponenty (src/components/ui)}

\subsection{ControlPanel}
\paragraph{Funkcia}
Panel pre pridávanie neurónov, bulk pridanie, spustenie/stop tréningu a reset.
Režim \texttt{mode} prepína tlačidlá Start/Stop a stavový indikátor.

\begin{lstlisting}[language=JavaScript, caption={ControlPanel -- rozhranie}, label={lst:ui-controlpanel}]
interface ControlPanelProps {
  mode: SimulationMode;
  onAddNeuron: (type: "input" | "hidden" | "output") => void;
  onAddMultiple?: (count: number, type: "input" | "hidden" | "output") => void;
  onStartTraining: () => void;
  onStopTraining: () => void;
  onReset: () => void;
  disabled?: boolean;
}
\end{lstlisting}

\subsection{AlgorithmPanel}
\paragraph{Funkcia}
Zobrazuje zoznam algoritmov z \texttt{ALGORITHMS}, umožňuje výber a spustenie.

\begin{lstlisting}[language=TypeScript, caption={AlgorithmPanel -- výber algoritmu}, label={lst:ui-algopanel}]
const [selectedAlgorithm, setSelectedAlgorithm] = useState<AlgorithmType | null>(null);
const handleRun = () => { if (selectedAlgorithm) onRunAlgorithm(selectedAlgorithm); };
\end{lstlisting}

\subsection{AlgorithmInfoOverlay}
\paragraph{Funkcia}
Overlay v 3D scéne s priebehom algoritmu, stavom a legendou.

\begin{lstlisting}[language=TypeScript, caption={Fázy overlayu}, label={lst:ui-overlay}]
useEffect(() => {
  if (!isRunning) setPhase('preparing');
  else if (neuronsCreated < totalNeurons) setPhase('creating');
  else if (progress < 0.99) setPhase('running');
  else setPhase('complete');
}, [isRunning, neuronsCreated, totalNeurons, progress]);
\end{lstlisting}

\subsection{StatsDisplay}
\paragraph{Funkcia}
Zobrazuje základné metriky, progress bary a upozornenia (overfitting/underfitting).

\begin{lstlisting}[language=TypeScript, caption={StatsDisplay -- progress bary}, label={lst:ui-stats}]
const ProgressBar = ({ value, color }: { value: number; color: string }) => (
  <div className="w-full h-2 bg-white/10 rounded-full overflow-hidden">
    <div className={`h-full ${color}`} style={{ width: `${value * 100}%` }} />
  </div>
);
\end{lstlisting}

\subsection{NeuronReferencePanel}
\paragraph{Funkcia}
Panel zobrazuje ``live'' najaktívnejší neurón a manuálne pripnutý neurón
so štatistikami (health, trainingCount, connections).

\begin{lstlisting}[language=TypeScript, caption={NeuronReferencePanel -- live focus}, label={lst:ui-reference}]
const selectedNeuron = neurons.find((n) => n.id === selectedNeuronId);
{liveFocus ? (
  <button onClick={() => onSelectNeuron(liveFocus.id)}>Pin live</button>
) : (
  <p>Spusti tréning alebo algoritmus...</p>
)}
\end{lstlisting}

\section{Odporúčané oddeľovanie blokov v zdrojových textoch}

\begin{lstlisting}[language=JavaScript, caption={Struktúra blokov pre Simulation moduly}, label={lst:sim-separators}]
/* ============================================================
   1) Importy + typy
   ============================================================ */

/* ------------------------------------------------------------
   2) Pomocné funkcie / konštanty
   ------------------------------------------------------------ */

/* ============================================================
   3) Hlavné komponenty / hooky
   ============================================================ */

/* ------------------------------------------------------------
   4) Side-effects (useEffect / intervaly / cleanup)
   ------------------------------------------------------------ */

/* ============================================================
   5) Exporty
   ============================================================ */
\end{lstlisting}

\end{document}
